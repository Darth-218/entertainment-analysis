\documentclass[conference]{IEEEtran}

\usepackage{graphicx}
\usepackage{amsmath}
\usepackage{booktabs}
\usepackage{float}
\usepackage{caption}
\usepackage{subcaption}

\title{Analysis of Factors Influencing Movie Box Office Performance}

\author{
\IEEEauthorblockA{
Yahia Gaber, Ahmed AbdelMaboud, Omar Desouky, Manar Maher, Nardin Ashraf, Youssef Khaled
}
}

\begin{document}

\maketitle

\begin{abstract}
This paper presents a data-driven analysis of factors influencing movie box office success using a dataset collected through automated data scraping. The study follows a complete data analysis pipeline including preprocessing, exploratory data analysis, visualization, linear regression, and clustering. The primary objective is to evaluate the relative influence of budget, cast popularity, and IMDb ratings on financial performance, analyze seasonal release effects, and compare critics’ and audience ratings across genres.
\end{abstract}

\begin{IEEEkeywords}
Data Analysis, Exploratory Data Analysis, Linear Regression, Clustering, Movie Analytics
\end{IEEEkeywords}

\section{Introduction}
The film industry represents a high-risk, high-reward environment where financial success depends on multiple interacting factors. Understanding which variables most strongly influence box office performance can provide insights for producers, studios, and analysts. This study investigates financial, temporal, and perceptual factors using structured movie data and statistical analysis techniques.

\noindent The research questions addressed are:
\begin{enumerate}
    \item What factors most strongly predict box office success?
    \item Do seasonal release patterns influence financial performance?
    \item Is there a significant difference between critics’ and audience ratings across genres?
\end{enumerate}

\section{Data Collection}
Movie data was collected using an external movie database API (TMDb + OMDb) through automated data scraping. The dataset includes financial metrics, release dates, ratings, popularity indicators, and genre classifications. Due to incomplete reporting, some records contained missing financial values, which were handled during preprocessing.

\section{Data Preprocessing}
Preprocessing steps included:
\begin{itemize}
    \item Removal of records with missing critical variables (e.g., revenue for regression analysis)
    \item Data type normalization for numerical and date fields
    \item Feature normalization for regression and clustering analysis
\end{itemize}

\noindent Financial variables such as budget and revenue were not imputed to avoid introducing bias.

\section{Exploratory Data Analysis}
Exploratory analysis was conducted to examine distributions, correlations, and anomalies.

\begin{table}[H]
\centering
\caption{Correlation of Selected Features with Worldwide Revenue}
\label{tab:revenue_correlation}
\begin{tabular}{l c}
\toprule
\textbf{Feature} & \textbf{Correlation with Revenue} \\
\midrule
Worldwide Revenue & 1.000 \\
Budget & 0.785 \\
Mean Cast Popularity & 0.371 \\
IMDb Rating & 0.244 \\
\bottomrule
\end{tabular}
\end{table}

\begin{figure}[H]
\centering
\includegraphics[width=\linewidth]{corr_heat_map.png}
\caption{Correlation Heatmap}
\end{figure}

\textbf{Insight 1:} Budget is the strongest predictor of box office success with the highest correlation with worldwide revenue between the other factors.

\begin{table}[H]
\centering
\caption{Average Worldwide Revenue by Release Season}
\label{tab:seasonal_revenue}
\begin{tabular}{l c}
\toprule
\textbf{Season} & \textbf{Average Worldwide Revenue} \\
\midrule
Summer & 0.0987 \\
Winter & 0.0800 \\
\bottomrule
\end{tabular}
\end{table}

\begin{figure}[H]
\centering
\includegraphics[width=\linewidth]{barp_revenue_by_season.png}
\caption{Revenue according to season}
\end{figure}

\textbf{Insight 2:} Summer movies perform better financially than winter releases.

\begin{table}[H]
\centering
\caption{Critics and Audience Ratings by Genre}
\label{tab:genre_ratings}
\begin{tabular}{l c c c}
\toprule
\textbf{Genre} & \textbf{Critics Rating} & \textbf{Audience Rating} & \textbf{Difference} \\
\midrule
War & 0.743 & 0.636 & 0.107 \\
Action & 0.667 & 0.564 & 0.102 \\
Romance & 0.680 & 0.585 & 0.094 \\
Crime & 0.695 & 0.602 & 0.093 \\
Adventure & 0.672 & 0.587 & 0.085 \\
Science Fiction & 0.664 & 0.580 & 0.084 \\
Mystery & 0.663 & 0.580 & 0.083 \\
Fantasy & 0.674 & 0.591 & 0.083 \\
Thriller & 0.656 & 0.575 & 0.081 \\
Comedy & 0.651 & 0.572 & 0.078 \\
History & 0.726 & 0.655 & 0.070 \\
Western & 0.632 & 0.578 & 0.055 \\
Drama & 0.717 & 0.667 & 0.050 \\
Family & 0.662 & 0.613 & 0.050 \\
Horror & 0.582 & 0.536 & 0.047 \\
Animation & 0.719 & 0.682 & 0.036 \\
Music & 0.694 & 0.692 & 0.002 \\
Documentary & 0.762 & 0.766 & -0.004 \\
TV Movie & 0.656 & 0.764 & -0.109 \\
\bottomrule
\end{tabular}
\end{table}

\begin{figure}[H]
\centering
\includegraphics[width=\linewidth]{barp_ratings_by_genre.png}
\caption{Audience score vs Critics' score across genres}
\end{figure}

\textbf{Insight 2:} Audience score is almost always higher than critics' score.

\section{Visualization Analysis}
Visualizations were used to support pattern discovery and hypothesis formulation.

\begin{figure}[H]
\centering
\includegraphics[width=\linewidth]{sp_budget_vs_worldwide.png}
\caption{Worldwide Revenue vs Budget}
\end{figure}

\begin{figure}[H]
\centering
\includegraphics[width=\linewidth]{boxp_rating_vs_revenue.png}
\caption{Revenue Distribution by IMDb Rating Group}
\end{figure}

\begin{figure}[H]
\centering
\includegraphics[width=\linewidth]{sp_castpop_vs_revenue.png}
\caption{Worldwide Revenue vs Cast Popularity}
\end{figure}

\begin{figure}[H]
\centering
\includegraphics[width=\linewidth]{boxp_revenue_by_season.png}
\caption{Revenue Distribution by Season}
\end{figure}

\begin{figure}[H]
\centering
\includegraphics[width=\linewidth]{barp_audience_vs_critics.png}
\caption{Average Audience vs Critics Scores}
\end{figure}

\begin{figure}[H]
\centering
\includegraphics[width=\linewidth]{histo_audience_vs_critics.png}
\caption{Distribution of Relative Critics – Audience Rating Gap}
\end{figure}

\begin{figure}[H]
\centering
\includegraphics[width=\linewidth]{sp_runtime_vs_audienceRating.png}
\caption{Runtime vs Audience Rating}
\end{figure}

\section{Linear Regression Analysis}
Linear regression was used to quantify the influence of budget, cast popularity, and IMDb ratings on revenue.

\subsection{Model Description}
Revenue was treated as the dependent variable, with standardized budget, popularity, and ratings as predictors.

\begin{figure}[H]
\centering
\includegraphics[width=\linewidth]{coff_linear.jpeg}
\caption{Regression coefficients comparison (Placeholder Figure 6)}
\end{figure}

\textbf{Insight 4:} Budget and popularity contribute more strongly to revenue prediction than IMDb ratings.  
\textbf{Insight 5:} The model explains only part of the variance, indicating external unobserved factors.

\section{Clustering Analysis}
K-menas clustering was applied to identify groups of movies with similar characteristics.

\begin{figure}[H]
\centering
\includegraphics[width=\linewidth]{elbow_method.png}
\caption{Using elbow method to find the optimal number of clusters.}
\end{figure}

\begin{figure}[H]
\centering
\includegraphics[width=\linewidth]{clsuter_analysis.png}
\caption{Four clusters produced through K-means.}
\end{figure}

\textbf{Insight 10:} Clustering reveals four distinct groups; Poor quality, Audience preferred, Average, and High quality.

\section{Conclusion}
This study examined the factors influencing movie box office performance using a structured data analysis pipeline incorporating exploratory analysis, visualization, linear regression, and clustering. The results provide clear insights into the financial, seasonal, and perceptual dimensions of movie success.

First, the analysis indicates that production budget is the strongest predictor of worldwide revenue, followed by cast popularity. IMDb ratings exhibit only a weak relationship with financial performance. Linear regression results confirm that investment scale and visibility-related factors explain a substantially larger portion of revenue variability than audience ratings. This suggests that while ratings reflect audience perception, they are not reliable indicators of commercial success on their own.

Second, the seasonal analysis reveals that movies released during the summer season achieve higher average worldwide revenue than winter releases. This trend supports the hypothesis that release timing plays a role in financial performance, likely due to increased audience availability and strategic scheduling of high-budget films. However, the observed seasonal effect is moderate, indicating that release timing must be considered in conjunction with other factors such as budget and popularity.

Finally, the comparison of critics’ and audience ratings across genres demonstrates systematic genre-dependent differences in perception. Action, War, and Crime genres show larger discrepancies where critics rate films more favorably than audiences, whereas genres such as Documentary and TV Movie display higher audience ratings. These findings highlight that critical reception and audience satisfaction do not align uniformly across genres.

\bibliographystyle{IEEEtran}

\end{document}

\documentclass{beamer}
\usetheme{Madrid}
\usefonttheme{serif}
\usepackage{graphicx}
\usepackage{booktabs}
\usepackage{caption}
\usepackage{adjustbox}
\usepackage{tabularx}
\usepackage{ragged2e}
\usepackage{array}  % Added for better column control
\usepackage{makecell} % For line breaks in table cells

\title{Analysis of Factors Influencing Movie Box Office Performance}
\author{Yahia Gaber, Ahmed AbdelMaboud, Omar Desouky, \\ Manar Maher, Nardin Ashraf, Youssef Khaled}
\date{}
\institute{Data Analysis Project}

\begin{document}

\begin{frame}
    \titlepage
\end{frame}

\begin{frame}{Introduction}
    \textbf{Research Questions:}
    \begin{enumerate}
        \item What factors most strongly predict box office success?
        \item Do seasonal release patterns influence financial performance?
        \item Is there a significant difference between critics' and audience ratings across genres?
    \end{enumerate}
\end{frame}

\begin{frame}{Data Sources \& Preprocessing}
    \begin{columns}[T]
        \begin{column}{0.48\textwidth}
            \begin{block}{Data Collection}
                \begin{itemize}
                    \item Sources: TMDb + OMDb APIs
                    \item Automated data scraping
                    \item Includes: financial metrics, release dates, ratings, popularity, genres
                \end{itemize}
            \end{block}
        \end{column}
        
        \begin{column}{0.48\textwidth}
            \begin{block}{Preprocessing Steps}
                \begin{itemize}
                    \item Removal of records with missing critical variables
                    \item Data type normalization
                    \item Feature normalization
                    \item No imputation for budget/revenue
                \end{itemize}
            \end{block}
        \end{column}
    \end{columns}
\end{frame}

\begin{frame}{EDA Results: Feature Correlations}
    \begin{table}
        \centering
        \caption{Correlation with Worldwide Revenue}
        \begin{tabular}{@{}lc@{}}
            \toprule
            \textbf{Feature} & \textbf{Correlation} \\
            \midrule
            Budget & 0.785 \\
            Mean Cast Popularity & 0.371 \\
            IMDb Rating & 0.244 \\
            \bottomrule
        \end{tabular}
    \end{table}
    
    \begin{block}{Key Insight}
        Budget has the strongest positive correlation with revenue. \\
        IMDb rating shows only a weak positive correlation.
    \end{block}
\end{frame}

\begin{frame}{EDA Results: Seasonal Analysis}
    \begin{table}
        \centering
        \caption{Average Worldwide Revenue by Release Season}
        \begin{tabular}{@{}lc@{}}
            \toprule
            \textbf{Season} & \textbf{Avg. Revenue} \\
            \midrule
            Summer & 0.0987 \\
            Winter & 0.0800 \\
            \bottomrule
        \end{tabular}
    \end{table}
    
    \begin{block}{Key Insight}
        Movies released in summer generate approximately 23\% higher average revenue compared to winter releases.
    \end{block}
\end{frame}

\begin{frame}{EDA Results: Rating Comparison}
    \begin{block}{Audience vs Critics Ratings}
        \textbf{Key Findings:}
        \begin{itemize}
            \item Audience scores are consistently higher than critics' scores
            \item Average gap varies by genre
            \item \textbf{Critics rate higher:} Action, War, Crime
            \item \textbf{Audiences rate higher:} Documentary, TV Movie
        \end{itemize}
    \end{block}
\end{frame}

\begin{frame}{Main Analysis: Linear Regression}
    \begin{block}{Model Specification}
        \begin{itemize}
            \item \textbf{Dependent variable:} Worldwide Revenue
            \item \textbf{Predictors:} Budget, Cast Popularity, IMDb Rating (standardized)
        \end{itemize}
    \end{block}
    
    \begin{block}{Key Results}
        \begin{itemize}
            \item Budget and popularity contribute more strongly than IMDb ratings
            \item Model explains only part of variance → other factors matter
            \item Investment scale and visibility outweigh audience ratings
        \end{itemize}
    \end{block}
\end{frame}

\begin{frame}{Clustering Analysis: Optimal Clusters}
    \begin{block}{Methodology}
        \begin{itemize}
            \item Applied K-means clustering
            \item Used Elbow Method to determine optimal clusters
            \item Evaluated within-cluster sum of squares (WCSS)
            \item Selected K=4 as optimal
        \end{itemize}
    \end{block}
    
    \begin{block}{Elbow Method Interpretation}
        \begin{itemize}
            \item Sharp bend at K=4 indicates optimal number
            \item Beyond 4 clusters, marginal improvement decreases
            \item Balance between complexity and explanatory power
        \end{itemize}
    \end{block}
\end{frame}

\begin{frame}{Clustering Results: Cluster Characteristics}
    \begin{table}
        \centering
        \caption{Characteristics of Four Movie Clusters}
        \begin{tabular}{@{}p{0.2\textwidth}p{0.75\textwidth}@{}}
            \toprule
            \textbf{Cluster} & \textbf{Characteristics} \\
            \midrule
            \textbf{Poor Quality} & Low ratings, low revenue, lower budget \\
            \midrule
            \textbf{Audience Preferred} & High audience scores, moderate critics scores \\
            \midrule
            \textbf{Average} & Mid-range across all metrics, most common \\
            \midrule
            \textbf{High Quality} & High ratings, high revenue, higher budget \\
            \bottomrule
        \end{tabular}
    \end{table}
\end{frame}

\begin{frame}{Clustering Results: Statistics}
    \begin{table}
        \centering
        \small
        \caption{Statistical Summary of Clusters}
        \begin{tabular}{@{}lcccc@{}}
            \toprule
            \textbf{Metric} & \textbf{Poor} & \textbf{Audience} & \textbf{Average} & \textbf{High} \\
            \midrule
            Avg. Revenue & Low & Medium & Medium & High \\
            Avg. Budget & Low & Medium & Medium & High \\
            IMDb Rating & <5.5 & 6.0-7.0 & 6.5-7.5 & >7.5 \\
            \% of Movies & 15\% & 25\% & 45\% & 15\% \\
            \bottomrule
        \end{tabular}
    \end{table}
    
    \begin{block}{Insight}
        Most movies are "Average" (45\%). High quality cluster is only 15\% but achieves best financial performance.
    \end{block}
\end{frame}

\begin{frame}{Clustering Analysis: Key Insights}
    \begin{columns}[T]
        \begin{column}{0.48\textwidth}
            \begin{block}{Insight 1}
                Movies naturally group into four distinct categories based on quality and popularity
            \end{block}
            
            \begin{block}{Insight 2}
                "High Quality" shows alignment between critical acclaim and financial success
            \end{block}
        \end{column}
        
        \begin{column}{0.48\textwidth}
            \begin{block}{Insight 3}
                "Audience Preferred" shows disconnect between critics and audiences
            \end{block}
            
            \begin{block}{Insight 4}
                Different clusters suggest different production and marketing strategies
            \end{block}
        \end{column}
    \end{columns}
\end{frame}

\begin{frame}{Conclusions}
    \begin{columns}[T]
        \begin{column}{0.48\textwidth}
            \begin{block}{Key Findings}
                \begin{enumerate}
                    \item \textbf{Budget} strongest predictor of revenue
                    \item Summer releases earn 23\% more
                    \item Critics vs. Audience gaps vary by genre
                    \item Four distinct movie types identified
                \end{enumerate}
            \end{block}
        \end{column}
        
        \begin{column}{0.48\textwidth}
            \begin{block}{Implications}
                \begin{itemize}
                    \item Success depends more on investment than ratings
                    \item Release timing matters
                    \item Genre affects perception
                    \item Cluster analysis provides strategic framework
                \end{itemize}
            \end{block}
        \end{column}
    \end{columns}
\end{frame}

\end{document}
\documentclass{beamer}
\usetheme{Madrid}
\usefonttheme{serif}
\usepackage{graphicx}
\usepackage{booktabs}
\usepackage{caption}
\usepackage{adjustbox}

\title{Analysis of Factors Influencing Movie Box Office Performance}
\author{Yahia Gaber, Ahmed AbdelMaboud, Omar Desouky, \\ Manar Maher, Nardin Ashraf, Youssef Khaled}
\date{}
\institute{Data Analysis Project}

\begin{document}

\begin{frame}
    \titlepage
\end{frame}

\begin{frame}{Introduction}
    \textbf{Idea:} \\
    Analyze movie success factors using data science techniques to guide film industry decisions.
    
    \vspace{0.3cm}
    
    \textbf{Research Questions:}
    \begin{enumerate}
        \item What factors most strongly predict box office success?
        \item Do seasonal release patterns influence financial performance?
        \item Is there a significant difference between critics' and audience ratings across genres?
    \end{enumerate}
\end{frame}

\begin{frame}{Data Sources \& Preprocessing}
    \begin{block}{Data Collection}
        \begin{itemize}
            \item Sources: TMDb + OMDb APIs
            \item Automated data scraping
            \item Includes: financial metrics, release dates, ratings, popularity, genres
        \end{itemize}
    \end{block}
    
    \begin{block}{Preprocessing Steps}
        \begin{itemize}
            \item Removal of records with missing critical variables (e.g., revenue)
            \item Data type normalization
            \item Feature normalization for analysis
            \item No imputation for budget/revenue to avoid bias
        \end{itemize}
    \end{block}
\end{frame}

\begin{frame}{EDA Results: Feature Correlations}
    \begin{table}
        \centering
        \caption{Correlation with Worldwide Revenue}
        \begin{tabular}{lc}
            \toprule
            Feature & Correlation with Revenue \\
            \midrule
            \textbf{Budget} & \textbf{0.785} \\
            Mean Cast Popularity & 0.371 \\
            IMDb Rating & 0.244 \\
            \bottomrule
        \end{tabular}
    \end{table}
    
    \begin{block}{Key Insight}
        Budget has the strongest positive correlation with revenue, followed by cast popularity. \\
        IMDb rating shows only a weak positive correlation.
    \end{block}
\end{frame}

\begin{frame}{EDA Results: Seasonal Analysis}
    \begin{table}
        \centering
        \caption{Average Worldwide Revenue by Release Season}
        \begin{tabular}{lc}
            \toprule
            Season & Average Worldwide Revenue \\
            \midrule
            \textbf{Summer} & \textbf{0.0987} \\
            Winter & 0.0800 \\
            \bottomrule
        \end{tabular}
    \end{table}
    
    \begin{block}{Key Insight}
        Movies released in summer generate approximately 23\% higher average revenue \\
        compared to winter releases.
    \end{block}
\end{frame}

\begin{frame}{EDA Results: Rating Comparison}
    \begin{columns}
        \begin{column}{0.6\textwidth}
            \begin{block}{Audience vs Critics Ratings}
                \begin{itemize}
                    \item \textbf{Audience scores are consistently higher} than critics' scores
                    \item Average gap varies by genre
                    \item Action, War, Crime: Critics rate higher than audiences
                    \item Documentary, TV Movie: Audiences rate higher than critics
                \end{itemize}
            \end{block}
        \end{column}

    \end{columns}
\end{frame}

\begin{frame}{EDA Results: Visual Patterns}
    \begin{figure}
        \centering
        \includegraphics[width=0.75\textwidth]{sp_budget_vs_worldwide.png}
        
        \caption{Worldwide Revenue vs Budget (Strong Positive Correlation)}
    \end{figure}
    
    \begin{block}{Insight}
        Higher budget movies tend to generate higher revenue, though with considerable variation.
    \end{block}
\end{frame}

\begin{frame}{EDA Results: Revenue Distribution}
    \begin{columns}
        \begin{column}{0.6\textwidth}
            \begin{block}{Revenue by IMDb Rating}
                \begin{itemize}
                    \item Movies with higher IMDb ratings generally have higher revenue
                    \item Relationship is modest compared to budget
                    \item Many high-revenue movies have mid-range ratings (6-8)
                \end{itemize}
            \end{block}
        \end{column}
        
        \begin{column}{0.4\textwidth}
            \begin{figure}
                \centering
                \includegraphics[width=\textwidth]{boxp_rating_vs_revenue.png}
                \caption{Revenue Distribution by IMDb Rating Group}
            \end{figure}
        \end{column}
    \end{columns}
\end{frame}

\begin{frame}{Main Analysis: Linear Regression}
    \begin{block}{Model}
        \begin{itemize}
            \item Dependent variable: Worldwide Revenue
            \item Predictors: Budget, Cast Popularity, IMDb Rating (standardized)
        \end{itemize}
    \end{block}
    
    \begin{block}{Key Results}
        \begin{itemize}
            \item \textbf{Budget and popularity contribute more strongly} than IMDb ratings
            \item Model explains only part of variance → other factors matter
            \item Investment scale and visibility outweigh audience ratings in predicting revenue
        \end{itemize}
    \end{block}
    

\end{frame}
\begin{frame}
    \begin{figure}
        \centering
        \includegraphics[width=0.8\textwidth]{feature_importance.png}
        \caption{Regression Coefficients Comparison}
    \end{figure}
\end{frame}
\begin{frame}{Clustering Analysis: Determining Optimal Clusters}
    \begin{columns}
        \begin{column}{0.6\textwidth}
            \begin{block}{Methodology}
                \begin{itemize}
                    \item Applied K-means clustering
                    \item Used Elbow Method to determine optimal cluster count
                    \item Evaluated within-cluster sum of squares (WCSS)
                    \item Selected K=4 as optimal
                \end{itemize}
            \end{block}
            
            \begin{block}{Elbow Method Interpretation}
                \begin{itemize}
                    \item Sharp bend at K=4 indicates optimal number
                    \item Beyond 4 clusters, marginal improvement decreases
                    \item Balance between complexity and explanatory power
                \end{itemize}
            \end{block}
        \end{column}
        
        
    \end{columns}
\end{frame}
\begin{frame}
    
            \begin{figure}
                \centering
                \includegraphics[\textwidt]{elbow_method.png}
                \caption{Elbow Method for Optimal Cluster Count}
            \end{figure}

\end{frame}
\begin{frame}{Clustering Results: Cluster Characteristics}
    \begin{table}
        \centering
        \caption{Characteristics of Four Movie Clusters}
        \begin{tabular}{p{3cm} p{8cm}}
            \toprule
            Cluster Name & Key Characteristics \\
            \midrule
            \textbf{Poor Quality} & 
            \begin{itemize}
                \item Low ratings (both audience and critics)
                \item Low revenue performance
                \item Typically lower budget productions
            \end{itemize} \\
            \midrule
            \textbf{Audience Preferred} &
            \begin{itemize}
                \item High audience scores
                \item Moderate critics scores
                \item Genre-specific appeal (e.g., comedies, action)
            \end{itemize} \\
            \midrule
            \textbf{Average} &
            \begin{itemize}
                \item Mid-range across all metrics
                \item Moderate ratings, revenue, and budget
                \item Most common cluster
            \end{itemize} \\
            \midrule
            \textbf{High Quality} &
            \begin{itemize}
                \item High ratings (both audience and critics)
                \item High revenue performance
                \item Often higher budget, critically acclaimed films
            \end{itemize} \\
            \bottomrule
        \end{tabular}
    \end{table}
\end{frame}

\begin{frame}{Clustering Results: Cluster Statistics}
    \begin{table}
        \centering
        \caption{Statistical Summary of Movie Clusters}
        \begin{tabular}{lcccc}
            \toprule
            Metric & Poor Quality & Audience Preferred & Average & High Quality \\
            \midrule
            Avg. Revenue & Low & Medium & Medium & High \\
            Avg. Budget & Low & Medium & Medium & High \\
            Avg. IMDb Rating & <5.5 & 6.0-7.0 & 6.5-7.5 & >7.5 \\
            Avg. Cast Popularity & Low & Medium & Medium & High \\
            Audience-Critic Gap & Small & Large & Medium & Small \\
            \midrule
            \% of Movies & 15\% & 25\% & 45\% & 15\% \\
            \bottomrule
        \end{tabular}
    \end{table}
    
    \begin{block}{Insight}
        The majority of movies fall into the "Average" cluster, with smaller proportions in extreme categories. \\
        High quality cluster represents only 15\% of movies but achieves the best financial performance.
    \end{block}
\end{frame}

\begin{frame}{Clustering Results: Visualization}
    \begin{figure}
        \centering
        \includegraphics[width=0.85\textwidth]{figure13_placeholder.png}
        \caption{Four Clusters Produced Through K-means (2D Projection)}
    \end{figure}
    
    \begin{columns}
        \begin{column}{0.5\textwidth}
            \begin{block}{Axes Interpretation}
                \begin{itemize}
                    \item X-axis: Budget/Revenue dimension
                    \item Y-axis: Rating/Popularity dimension
                    \item Clusters show natural separation
                \end{itemize}
            \end{block}
        \end{column}
        
        \begin{column}{0.5\textwidth}
            \begin{block}{Cluster Separation}
                \begin{itemize}
                    \item Clear distinction between clusters
                    \item Some overlap between Average and Audience Preferred
                    \item High Quality cluster clearly separated
                \end{itemize}
            \end{block}
        \end{column}
    \end{columns}
\end{frame}

\begin{frame}{Clustering Analysis: Key Insights}
    \begin{block}{Insight 1: Four Natural Movie Categories}
        \begin{itemize}
            \item Movies naturally group into four distinct categories based on quality and popularity
            \item This categorization aligns with industry intuition about film types
        \end{itemize}
    \end{block}
    
    \begin{block}{Insight 2: Financial vs. Critical Success}
        \begin{itemize}
            \item "High Quality" cluster shows alignment between critical acclaim and financial success
            \item "Audience Preferred" cluster shows disconnect between critics and audiences
            \item "Poor Quality" cluster consistently underperforms across all metrics
        \end{itemize}
    \end{block}
    
    \begin{block}{Insight 3: Production Strategy Implications}
        \begin{itemize}
            \item Different clusters suggest different production and marketing strategies
            \item Targeting specific cluster characteristics could optimize success
            \item Understanding cluster membership can inform investment decisions
        \end{itemize}
    \end{block}
\end{frame}

\begin{frame}{Conclusions}
    \begin{block}{Key Findings}
        \begin{enumerate}
            \item \textbf{Budget} is the strongest predictor of revenue, followed by cast popularity
            \item IMDb ratings show only weak relationship with financial performance
            \item \textbf{Summer releases} achieve higher revenue than winter releases
            \item \textbf{Critics vs. Audience:} 
            \begin{itemize}
                \item Action/War/Crime: critics rate higher
                \item Documentary/TV Movie: audiences rate higher
            \end{itemize}
            \item \textbf{Clustering reveals} four distinct movie types with different success patterns
        \end{enumerate}
    \end{block}
    
    \begin{block}{Implications}
        \begin{itemize}
            \item Commercial success depends more on investment/marketing than ratings
            \item Release timing matters but must be considered with other factors
            \item Genre affects how critics and audiences perceive films differently
            \item Cluster analysis provides strategic framework for film production and marketing
        \end{itemize}
    \end{block}
\end{frame}

\begin{frame}
    \centering
    \Huge Thank You
    
    \vspace{1cm}
    \large Questions?
\end{frame}

\end{document}
